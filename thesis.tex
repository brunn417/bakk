% Copyright (C) 2014-2020 by Thomas Auzinger <thomas@auzinger.name>

\documentclass[draft,final]{vutinfth} % Remove option 'final' to obtain debug information.

% Load packages to allow in- and output of non-ASCII characters.
\usepackage{lmodern}        % Use an extension of the original Computer Modern font to minimize the use of bitmapped letters.
\usepackage[T1]{fontenc}    % Determines font encoding of the output. Font packages have to be included before this line.
\usepackage[utf8]{inputenc} % Determines encoding of the input. All input files have to use UTF8 encoding.

% Extended LaTeX functionality is enables by including packages with \usepackage{...}.
\usepackage{csquotes}   % used for easy quotes
\usepackage{amsmath}    % Extended typesetting of mathematical expression.
\usepackage{amssymb}    % Provides a multitude of mathematical symbols.
\usepackage{mathtools}  % Further extensions of mathematical typesetting.
\usepackage{microtype}  % Small-scale typographic enhancements.
\usepackage[inline]{enumitem} % User control over the layout of lists (itemize, enumerate, description).
\usepackage{multirow}   % Allows table elements to span several rows.
\usepackage{booktabs}   % Improves the typesettings of tables.
\usepackage{subcaption} % Allows the use of subfigures and enables their referencing.
\usepackage[ruled,linesnumbered,algochapter]{algorithm2e} % Enables the writing of pseudo code.
\usepackage[usenames,dvipsnames,table]{xcolor} % Allows the definition and use of colors. This package has to be included before tikz.
\usepackage{nag}       % Issues warnings when best practices in writing LaTeX documents are violated.
\usepackage{todonotes} % Provides tooltip-like todo notes.
\usepackage{hyperref}  % Enables cross linking in the electronic document version. This package has to be included second to last.
\usepackage[acronym,toc]{glossaries} % Enables the generation of glossaries and lists fo acronyms. This package has to be included last.

\usepackage{listings}

% Definieren der Sprache GraphQL für lstlisting
\lstdefinelanguage{GraphQL}{
	keywords={query, mutation, subscription, true, false, null},
	keywordstyle=\color{blue}\bfseries,
	ndkeywords={type, String, Int, Boolean, ID},
	ndkeywordstyle=\color{darkgray}\bfseries,
	identifierstyle=\color{black},
	sensitive=false,
	comment=[l]{\#},
	morestring=[b]",
	morestring=[b]',
}

\lstset{
	language=GraphQL,
	showstringspaces=false,
	extendedchars=true,
	basicstyle=\ttfamily\small,
	showspaces=false,
	showtabs=false,
	frame=single,
	tabsize=2,
	breaklines=true,
	showstringspaces=false
}

% Define convenience functions to use the author name and the thesis title in the PDF document properties.
\newcommand{\authorname}{Paul Brunner} % The author name without titles.
\newcommand{\thesistitle}{} % The title of the thesis. The English version should be used, if it exists.

% Set PDF document properties
\hypersetup{
    pdfpagelayout   = TwoPageRight,           % How the document is shown in PDF viewers (optional).
    linkbordercolor = {Melon},                % The color of the borders of boxes around crosslinks (optional).
    pdfauthor       = {\authorname},          % The author's name in the document properties (optional).
    pdftitle        = {\thesistitle},         % The document's title in the document properties (optional).
    pdfsubject      = {Subject},              % The document's subject in the document properties (optional).
    pdfkeywords     = {a, list, of, keywords} % The document's keywords in the document properties (optional).
}

\setpnumwidth{2.5em}        % Avoid overfull hboxes in the table of contents (see memoir manual).
\setsecnumdepth{subsection} % Enumerate subsections.

\nonzeroparskip             % Create space between paragraphs (optional).
\setlength{\parindent}{0pt} % Remove paragraph identation (optional).

\makeindex      % Use an optional index.
\makeglossaries % Use an optional glossary.
%\glstocfalse   % Remove the glossaries from the table of contents.

% Set persons with 4 arguments:
%  {title before name}{name}{title after name}{gender}
%  where both titles are optional (i.e. can be given as empty brackets {}).
\setauthor{}{\authorname}{}{}
\setauthorextra
\setadvisor{Dipl.-Ing.}{Thomas Artner}{}{male}

% For bachelor and master theses:
%\setfirstassistant{Pretitle}{Forename Surname}{Posttitle}{male}
%\setsecondassistant{Pretitle}{Forename Surname}{Posttitle}{male}
%\setthirdassistant{Pretitle}{Forename Surname}{Posttitle}{male}

% For dissertations:
\setfirstreviewer{Pretitle}{Forename Surname}{Posttitle}{male}
\setsecondreviewer{Pretitle}{Forename Surname}{Posttitle}{male}

% For dissertations at the PhD School and optionally for dissertations:
\setsecondadvisor{Pretitle}{Forename Surname}{Posttitle}{male} % Comment to remove.

% Required data.
\setregnumber{11919163}
\setdate{01}{06}{2024} % Set date with 3 arguments: {day}{month}{year}.
\settitle{\thesistitle}{Die Evaluierung und Implementierung einer Web-API} % Sets English and German version of the title (both can be English or German). If your title contains commas, enclose it with additional curvy brackets (i.e., {{your title}}) or define it as a macro as done with \thesistitle.
\setsubtitle{Optional Subtitle of the Thesis}{Technologieevaluierung und Implementierung einer Web-
		Schnittstelle für Gebäudezertifizierungsdaten:
		Eine Fallstudie bei der Pulswerk GmbH.} % Sets English and German version of the subtitle (both can be English or German).

% Select the thesis type: bachelor / master / doctor / phd-school.
% Bachelor:
\setthesis{bachelor}
%
% Master:
%\setthesis{master}
%\setmasterdegree{dipl.} % dipl. / rer.nat. / rer.soc.oec. / master
%
% Doctor:
%\setthesis{doctor}
%\setdoctordegree{rer.soc.oec.}% rer.nat. / techn. / rer.soc.oec.
%
% Doctor at the PhD School
%\setthesis{phd-school} % Deactivate non-English title pages (see below)

% For bachelor and master:
\setcurriculum{Media Informatics and Visual Computing}{Software \& Information Engineering} % Sets the English and German name of the curriculum.

% For dissertations at the PhD School:
\setfirstreviewerdata{Affiliation, Country}
\setsecondreviewerdata{Affiliation, Country}


\begin{document}

\frontmatter % Switches to roman numbering.
% The structure of the thesis has to conform to the guidelines at
%  https://informatics.tuwien.ac.at/study-services

%\addtitlepage{naustrian} % German title page (not for dissertations at the PhD School).
% \addtitlepage{english} % English title page.
\addinsotitlepage{naustrian}
\addstatementpage

\begin{danksagung*}
\todo{Ihr Text hier.}
danke mutti
\end{danksagung*}

\begin{acknowledgements*}
\todo{Enter your text here.}
wtf ist das
\end{acknowledgements*}

\begin{kurzfassung}
\todo{Ihr Text hier.}
Dieses Template dient als Vorlage für die Erstellung einer Diplomarbeit am INSO. Individuelle Erweiterungen, Strukturanpassungen und Layout-Veränderungen können und sollen selbstverständlich nach persönlichem Ermessen und in Rücksprache mit Ihrem Betreuer vorgenommen werden. 

Diplomarbeiten aus Informatik können in deutscher oder englischer Sprache verfasst werden, Arbeiten aus Business Informatics müssen auf Englisch geschrieben werden.

Die Kurzfassung ist der Teil der Arbeit, der wohl am häufigsten gelesen wird – so wird sie beispielsweise im Epilog-Band der Fakultät publiziert und einem breiten Publikum verfügbar gemacht. Empfohlen wird, die Kurzfassung erst nach Finalisierung der gesamten Arbeit zu schreiben.

Aufbau: In der Kurzfassung werden auf einer 3/4 bis maximal einer Seite die Kernaussagen der Diplomarbeit zusammengefasst. Dabei sollte zunächst die Motivation/ der Kontext der vorliegenden Arbeit dargestellt werden, und dann kurz die Frage-/ Problemstellung erläutert werden, max. 1 Absatz! Im nächsten Absatz auf die Methode/ Verfahrensweise/ das konkrete Fallbeispiel eingehen, mit deren Hilfe die Ergebnisse erzielt wurden. Im Zentrum der Kurzfassung stehen die zentralen eigenen Ergebnisse der Arbeit, die den Wert der vorliegenden wissenschaftlichen Arbeit ausmachen. Hier auch, wenn vorhanden, eigene Publikationen erwähnen.

Wichtig: Verständlichkeit! Abkürzungen immer zuerst ausschreiben, in Klammer die Erklärung: Im Rahmen der vorliegenden Arbeit werden Non Governmental-Organisationen (NGOs) behandelt, \ldots

Bei theoretischen Diplomarbeiten, z.B. Literaturüberblick und Grundlagen zu einem größeren Themenblock, sollte in der Kurzfassung deutlich der Bedarf an einer solchen Übersicht und der Nutzen für die akademische Gemeinschaft aufgezeigt werden.


\textbf{Keywords:} \emph{5 – max. 8 Keywords zur Arbeit eingeben}
\end{kurzfassung}

\begin{abstract}
\todo{Enter your text here.}
Hier werden auf einer halben bis maximal einer Seite die Kernaussagen der Diplomarbeit in Englisch zusammengefasst ( = Übersetzung der Kurzfassung, am besten von einem \enquote{Native Speaker} Korrektur lesen lassen). Englischer Abstract ist auch bei auf Deutsch geschriebenen Arbeiten verpflichtend von der Fakultät vorgesehen.


\textbf{Keywords:} \emph{Übersetzung der deutschen Keywords}
\end{abstract}

% Select the language of the thesis, e.g., english or naustrian.
\selectlanguage{naustrian}

% Add a table of contents (toc).
\tableofcontents % Starred version, i.e., \tableofcontents*, removes the self-entry.

% Switch to arabic numbering and start the enumeration of chapters in the table of content.
\mainmatter


\chapter{EINLEITUNG}

\section{Hintergrund und Motivation}

Die Pulswerk GmbH ist seit mehr als 2 Jahren mein Arbeitgeber.
Ihr Tätigkeitsbereich ist sehr umfassend, aber einer der Kernbereiche ist die Bereitstellung von Web-Applikationen zur Zertifizierung von Gebäudedaten.
Für verschiedene Zertifikate und Kunden gibt es unterschiedliche Plattformen und Kriteriensets, jedoch ist eine generische Grundstruktur die Basis. 
Mein derzeitiger Tätigkeitsbereich ist die Qualitätssicherung und Unterstützung bei der Weiterentwicklung dieser Plattformen.
Diese Gebäude-spezifischen Daten sind auf dem Webserver der Pulswerk GmbH gespeichert und nur über diese Formulare und bestimmten Webseiten zugänglich.
Damit ein strukturierter und übersichtlicher Zugang zu diesen Daten ermöglicht wird, soll eine Web-Schnittstelle in Form eines Application Programming Interface (API) implementiert werden.

Im Zuge dieser Bachelorarbeit für Informatik, die zum Abschluss des Bachelorstudiums Software \& Information 033 534 Engineering an der Technische Universität Wien Voraussetzung ist, werde ich diese Problemstellung behandeln und zu einer Lösung beitragen.
 
 
\section{Zielsetzung der Arbeit}

Da es in der Zukunft viele unterschiedliche Benutzergruppen dieser Schnittstelle geben soll, wird eine generische, aber auf die Anforderungen abgestimmte, Implementierung erwartet. 
Um dies zu ermöglichen, muss im Vorfeld der Implementierung eine genaue Evaluierung der verschiedenen Technologien und Ansprüche durchgeführt werden.
 
Das Ziel dieser Arbeit ist es, den derzeitigen Stand der Technik in Bezug auf Web-Schnittstellen zu recherchieren und auf dem erlangten Wissen aufbauend eine API entwerfen.
Diese soll nicht nur den Ansprüchen der Stakeholder gerecht werden, sondern auch technische nicht funktionalen Anforderungen sättigen.
Dabei sind Aspekte wie Sicherheit, Antwortzeit, Benutzerfreundlichkeit und Skalierbarkeit grundlegende Kriterien, aber auch Parameter wie Over- und Underfetching
\footnote{
	Underfetching tritt auf, wenn eine Schnittstellenabfrage nicht alle erwünschten Daten in der Antwort enthält wohingegen beim Overfetching mehr als das Angeforderte retourniert wird.
} 
sind von Bedeutung und werden bei der Anforderungsanalyse berücksichtigt. 

Mittels Continuous Deployment soll eine nahtlose Integration in die bestehende Systemlandschaft der Pulswerk GmbH ermöglicht werden. 
Dies gewährleistet, dass die Web-Schnittstelle nicht nur den momentanen Anforderungen gerecht wird, sondern auch entsprechend flexibel ist, um sich an zukünftige Entwicklungen anzupassen.


\section{Struktur der Arbeit}

Zum Beginn der Arbeit werden mögliche Anwendungsfälle simuliert und in Zusammenarbeit mit den Stakeholder eine Anforderungsanalyse durchgeführt. 
Dabei werden die funktionalen und nicht-funktionalen Anforderungen untersucht und festgelegt.
Die daraus resultierenden Leistungsanforderungen liefern das Grundkonzept für die Schnittstelle. 

Im nächsten Schritt wird eine umfassende Recherche des aktuellen Stands der Technik und dazugehörenden Literatur durchgeführt. 
Hier liegt der Fokus auf der Auswahl welches Entwurfsmuster für diesen Anwendungsfall herangezogen werden soll und mithilfe welcher Technologien die Implementierung umzusetzen ist. 

Nach der Technologieevaluierung wird eine prototypische Implementierung gemacht. 
Dabei werden relevante Technologien und Frameworks entsprechend den Projektanforderungen ausgewählt. 
Dieser Prototyp dient als Grundlage für die iterative Verbesserung und Entwicklung einer vollständigen Web-Schnittstelle.

Nach der prototypischen Implementierung wird abschließend eine Validierung und Einschätzung der entwickelten Web-Schnittstelle durch Experteninterviews durchgeführt. 
Ziel ist es, Fachwissen und Erfahrung externer und interner Experten zu nutzen, um die entwickelte Lösung kritisch zu bewerten und mögliche Schwachstellen oder Optimierungspotenziale aufzudecken.










\chapter{GRUNDLAGEN}

Im Grundlagenkapitel wird zunächst das Beratungsunternehmen Pulswerk GmbH vorgestellt, gefolgt von einer Betrachtung der wesentlichen Web-Anwendungen. Anschließend wird die Notwendigkeit einer Web-API für Datenzugriff erläutert. 
Im weiteren Verlauf werden theoretische Konzepte und Designprinzipien von Web-Schnittstellen behandelt, die für die Erstellung benutzerfreundlicher und zukunftsfähiger APIs erforderlich sind.
Abschließend werden die Anforderungen an die Schnittstelle definiert, die aus einer detaillierten Analyse resultieren. 


\section{Beschreibung des Unternehmens und der Webapplikationen}

Die Pulswerk GmbH, ein Tochterunternehmen des Österreichischen Ökologie-Instituts, hat sich seit ihrer Gründung 1985 als ein Akteur im Bereich der ökologischen und nachhaltigen Entwicklung etabliert. 
Das Unternehmen tritt in einer breiten Palette von Sektoren als Initiator und Mitgestalter für zukunftsfähige Projekte auf und engagiert sich in diversen Netzwerken für die Planung und Realisierung nachhaltiger Initiativen.

Ein spezifischer Fokus der Pulswerk GmbH liegt auf dem Baugewerbe, insbesondere in Bezug auf Herausforderungen des Klimaschutzes. 
Sie bietet einige Webapplikationen an, die es ermöglichen, Gebäude nach verschiedenen Richtlinien wie klimaaktiv
\footnote{
	klimaaktiv ist ein österreichisches Programm zur Förderung klimafreundlichen Bauens, unterstützt durch das Bundesministerium für Klimaschutz. 
	Es bietet unter Anderem Zertifizierungen für energieeffiziente und nachhaltige Gebäude.
} 
oder EU-Taxonomie
\footnote{
	Die EU-Taxonomie ist ein EU-Klassifikationssystem, das festlegt, welche Investitionen als ökologisch nachhaltig
	gelten, um grüne Investitionen zu unterstützen und den Klimaschutz voranzutreiben.
	Die EU-Taxonomie bezieht sich auf alle Wirtschaftssektoren, wobei sich die Pulswerk GmbH mit dem Bausektor befasst. 
} 
einfach und ohne bürokratischen Aufwand zu zertifizieren. 

Diese Kataloge, die als umfangreiche Formulare mit diversen Eingabeoptionen wie Textfeldern, Checkboxen, Radio-Buttons und Upload-Funktionen konzipiert sind, erlauben Nutzern, Daten einzutragen und abhängig von Plattform und spezifischem Kriterienkatalog Punkte zu erhalten.
Innerhalb der verschiedenen Plattformen kommen unterschiedliche Kriterienkataloge zum Einsatz, die sich nicht nur zwischen den einzelnen Webapplikationen unterscheiden, sondern auch innerhalb einer Plattform abhängig von der Version, dem Gebäudetyp und anderen Faktoren variieren können. 
Zusätzlich ist es möglich, sofern es konfiguriert wurde, Projekte auch zwischen verschiedenen Kriterienkataloge zu verschieben, und die übereinstimmenden Felder werden übernommen. 
Somit können Projekte gleichzeitig für mehrere Zertifizierungen deklariert und eingereicht werden.

Sobald ein Projekt alle Mindestanforderungen erfüllt und die Eingaben komplettiert sind, kann es zur Plausibilitätsprüfung eingereicht werden. 
Diese Prüfung wird von internen oder externen Auditoren vorgenommen und kann sich über mehrere Durchläufe erstrecken. 
Nach erfolgreichem Abschluss der Überprüfung besteht die Option, das zertifizierte Gebäude auf verschiedenen Webseiten visuell ansprechend aufbereitet zu präsentieren und damit für die Öffentlichkeit zugänglich zu machen.


\section{Erfordernis einer API für den internen und externen Datenzugriff}

In der aktuellen Dateninfrastruktur der Plattformen befinden sich über 10.000 Gebäudedatensätze, die entweder noch bearbeitet oder bereits finalisiert sind. 
Die dezentrale Verteilung dieser Datensätze schränkt jedoch die Möglichkeit eines kontrollierten Zugriffs ein. 
Daraus ergibt sich die Notwendigkeit einer Schnittstelle, die in der Lage ist, effizient durch diese umfangreichen Datenbestände zu navigieren und die geforderten Informationen abzurufen. 
Diese Schnittstelle ermöglicht somit einen systematischen und strukturierten Zugang zu den in Datenbanken auf dem Server gespeicherten Datensätzen und trägt damit zur Optimierung der Datenverwaltung und Datennutzung bei.

Von solch einer Schnittstelle profitieren sowohl interne Teams als auch externe Kunden. 
Für die internen Operationen der Pulswerk GmbH ermöglicht die Schnittstelle eine vertiefte Analyse und Evaluation der durchgeführten Arbeiten, was zu einer verbesserten Effizienz und Effektivität der internen Prozesse beitragen kann. 
Externe Kunden erhalten durch die API einen detaillierten Einblick in ihre individuellen Beiträge und den jeweiligen Fortschritt in Bezug auf Klimaschutzmaßnahmen. 
Zusätzlich unterstützt die Integration der API eine nahtlose Kommunikation mit bestehenden Abrechnungssystemen, wie z.B. SAP
\footnote{
	Systems, Applications \& Products in Data Processing ist ein weltweit führendes Unternehmen im Bereich Unternehmenssoftware. 
	Diese Software unterstützt Firmen bei der Verwaltung und Optimierung ihrer Geschäftsprozesse.
}
, was eine Vereinfachung der Geschäfts- und Abrechnungsprozesse zur Folge hat.
Den Kunden wird über diese Plattform Zugang gewährt, wobei der Zugriff durch entsprechende Berechtigungen reguliert wird. 
Die Entwicklung einer generischen API, die diesen Anforderungen gerecht wird, sowie die Erstellung einer nachvollziehbaren Dokumentation bildet die essentielle Basis, um Benutzerfreundlichkeit des Systems sicherzustellen.


\section{Theoretische Konzepte von Web-Schnittstellen}

APIs sind essentielle Mechanismen, die Interaktionen zwischen zwei Softwarekomponenten ermöglichen. 
Beide Seiten verwenden dafür bestimmte Schnittstellenspezifikationen und Protokolle, um einen nahtlosen Datenaustausch zu gewährleisten, oder im Fehlerfall aussagekräftige Fehlermeldungen bereitzustellen.
Es existiert eine Vielzahl an Architekturen solcher Schnittstellen, die je nach Anwendungsfall eingesetzt werden.
Im Grundsatz sendet eine Client-Applikation eine Anfrage, die über ein Netzwerk an den Server gerichtet wird. 
Dieser interpretiert die Anfrage gemäß der API-Konventionen, verarbeitet diese, und antwortet dem Client mit den angefragten Daten.

Web-Schnittstellen repräsentieren eine spezialisierte Kategorie von APIs, die spezifisch für eine Kommunikation zwischen Web-Clients –- üblicherweise Web-Browsern –- und Web-Servern ausgelegt sind. 
Hierbei wird der Datenaustausch über das Hypertext Transfer Protocol (HTTP) oder dessen sichere Variante, HTTPS, abgewickelt, wobei eine große Bandbreite an Datenformaten wie HTML, XML oder JSON zum Einsatz kommen kann.


\section{Designprinzipien von Web-Schnittstellen}

Die Designs von APIs unterliegen einer Vielzahl von Ansätzen, deren übergreifendes Ziel es ist, eine robuste und intuitiv bedienbare Schnittstelle zu entwickeln. 
Im Zentrum steht der Entwurf einer API, die nicht nur in ihrer Lebensdauer beständig, sondern auch wartungsarm ist. 
Eine effektive API zeichnet sich durch ihre Benutzerfreundlichkeit aus und bedarf keiner umfangreichen Dokumentation, da sie weitgehend selbsterklärend sein soll \cite{Bloch:2006:How2DesignGoodAPI}.

Im Vorfeld der Entwicklung einer solchen Schnittstelle ist eine umfassende Anforderungsanalyse essenziell, um eine adäquate Erfüllung der spezifischen Ansprüche zu gewährleisten. 
Mit der Weiterentwicklung der Anwendungsfälle muss auch die API in ihrer Funktionalität flexibel und adaptiv gestaltet sein, um zukünftigen Anforderungen gerecht zu werden. 
Dies bedingt, dass die Weiterentwicklung der API stets unter Berücksichtigung dieser dynamischen Entwicklung erfolgt. \cite{Bloch:2006:How2DesignGoodAPI}

Des Weiteren spielen aussagekräftige Beispiele für die Nutzung der API eine grundlegende Rolle, um deren Leistungsfähigkeit und Anwendbarkeit zu veranschaulichen. 
Die Klarheit in der Namensgebung ist ein weiterer kritischer Faktor, der die Zugänglichkeit und Verständlichkeit der API maßgeblich beeinflusst. Darüber hinaus ist es vorteilhaft, wenn alle Anfragen im String-Format verarbeitet werden können und die API konsistente sowie typgerechte Rückgabeformate liefert, um die Datenverarbeitung und Datenintegration zu vereinfachen. \cite{Bloch:2006:How2DesignGoodAPI}


\section{Anforderungen an die Schnittstelle}

mach ich wenn die anforderungsanalyse erfolgt ist












\chapter{ÄHNLICHE ARBEITEN}

In den letzten Jahren hat der Bereich der Web-Schnittstellen, gesteuert durch die Zunahme an Webanwendungen, eine dynamische Entwicklung erfahren. Diese Entwicklung wird durch eine Vielzahl von wissenschaftlichen Arbeiten unterstrichen, die sich mit den verschiedenen Aspekten und Herausforderungen von Web-APIs auseinandersetzen. 
Um einen fundierten Überblick über den aktuellen Wissensstand im Bereich der Web-APIs zu erhalten, ist eine Analyse bestehender Forschungsarbeiten notwendig.  
Im weiteren Verlauf dieser Arbeit liegt der Fokus auf der Untersuchung spezifischerer Themengebiete, um eine Technologieevaluierung durchführen zu können. 
Dabei werden insbesondere die Vor- und Nachteile verschiedener Technologien im Kontext der Entwicklung und Implementierung von Web-Schnittstellen beleuchtet.


\section{Web-APIs}

Die Nutzung von Web-Schnittstellen birgt Herausforderungen, insbesondere aufgrund des Fehlens einheitlicher Standards \cite{Alrashed:2021:StandardizingAPIs}. 
Obwohl viele Webseiten Dritten den Zugang zu ihren Daten über Web-APIs gewähren, stellt die manuelle Erstellung von URLs -- einschließlich der Festlegung von Endpunkten, Parametern, Authentifizierungsprozessen und der Handhabung von Seitenumbrüchen -- eine erhebliche Hürde dar 
\cite{Alrashed:2021:StandardizingAPIs, Maleshkova:2010:InvestigationWebAPIs}. 
Diese Prozesse erfordern ein tiefgehendes Verständnis und sind mit einem beträchtlichen Zeitaufwand verbunden \cite{Alrashed:2021:StandardizingAPIs}. 
Zudem verfügt eine große Anzahl dieser Web-APIs über individuelle Datenmodelle, die, obwohl sie häufig ähnliche Arten von Informationen verwalten, unterschiedliche Methoden und Eigenschaften auf ihre eigene Weise bereitstellen \cite{Alrashed:2021:StandardizingAPIs}.
Um sich Zugang zu diesen Systemen zu ermöglichen, ist es notwendig, sich mit diesen im Voraus vertraut zu machen. 
Jedoch wird dies durch die Vielfalt und oft inkonsistente Art der Dokumentation zusätzlich erschwert \cite{Maleshkova:2010:InvestigationWebAPIs}.

Eine umfangreiche empirische Studie von S. Serbout,  F. D. Lauro und C. Pautasso, die auf der IEEE International Conference on Software Architecture Workshops 2022 vorgestellt wurde, untersucht die Größe und Struktur von Web-API-Spezifikationen, die aus Open-Source-Repositories extrahiert wurden. Diese Web-APIs umfassen neben der Dokumentation der bereitgestellten Operationen auch Schemadefinitionen für den Datenaustausch bei API-Anfragen und API-Antworten. 
Durch die Analyse von 42.194 gültigen OAS-Spezifikationen, die zwischen 2014 und 2021 veröffentlicht wurden und von bekannten Dienstanbietern wie Google, VMware (Avi Networks), Twilio und Amazon stammen, zeigt die Studie verschiedene Metriken auf. 
Der Mittelwert über alle diese Schnittstellen ist 3,98 Pfade zu 5,23 Operationen. 
Außerdem wurde eine Korrelation zwischen der Größe der API und Art des Designkonzeptes entdeckt. 
So sind kleine Schnittstellen meißt nur Read-Only und in Remote Procedure Call (RPC) Stil implementiert. 
Wohingegen Schnittstellen mit drei oder mehr Pfade in Richtung Create, Read, Update and Delete (CRUD) oder Representational State Transfer (REST) tendieren. \cite{Serbout:2022:WebApiStructures}


\section{REST}

%vlt kurz rest allgemein beschreiben wie unten graphql + person id beispiel

Obwohl viele Web-Schnittstellen als REST-APIs bezeichnet werden, erfüllen sie selten vollständig die in Roy Fieldings Dissertation definierten REST-Prinzipien \cite{Neumann:2021:AnalysisOfRest}. 
In Wahrheit implementieren die meisten dieser Schnittstellen nur einige Aspekte von REST, ohne jedoch den vollständigen Architekturstil
\footnote{
	Eine Softwarearchitektur umfasst die strukturierten Komponenten eines Softwaresystems, die in Abhängigkeit zueinander stehen und miteinander kommunizieren müssen, um eine funktionierende Anwendung zu ermöglichen \cite{Fielding:2000:REST}. 
	Der Architekturstil legt fest, welche architektonischen Voraussetzungen erfüllt sein müssen, damit ein System diesem Stil zugeordnet werden kann.
}
zu erreichen, den Fielding beschrieben hat \cite{Neumann:2021:AnalysisOfRest}.

Das REST-Paradigma definiert sechs spezifische Einschränkungen, wovon fünf eingehalten werden müssen, um als RESTful zu gelten. 
Diese Prinzipien sind entscheidend, um die Vorteile von REST, wie Skalierbarkeit, Zustandslosigkeit und die Fähigkeit, unabhängige Dienste zu nutzen, vollständig auszuschöpfen. 
Leider zeigt die Praxis, dass viele als REST deklarierte Schnittstellen diesen Anforderungen nicht gerecht werden, was oft zu Missverständnissen und ineffizienter Implementierung führt \cite{Neumann:2021:AnalysisOfRest}.

\subsection{Client-Server}

Die Client-Server-Abstraktion ist eine essenzielle Einschränkung im REST-Architekturstil, die die Benutzerschnittstelle (Client) von der Datenspeicherung und Geschäftslogik (Server) separiert. 
Diese Trennung fördert die Möglichkeit, dass Benutzer über verschiedene Plattformen auf Anwendungen zugreifen können, während Serverfunktionen wie Datenmanagement und Verarbeitungslogik unabhängig weiterentwickelt und skaliert werden. 
Dadurch wird nicht nur die Wartbarkeit und Anpassungsfähigkeit der Systeme erhöht, sondern auch eine Plattformunabhängigkeit gewährleistet, die eine breite Nutzerbasis anspricht. 
Diese modulare Struktur unterstützt die Skalierbarkeit, indem sie die Verteilung von Lasten und die effiziente Ressourcennutzung ohne direkte Abhängigkeiten zwischen Client und Server ermöglicht, wie Fielding in seiner Arbeit „Architectural Styles and the Design of Network-based Software Architectures“ hervorhebt. \cite{Fielding:2000:REST}.


\subsection{Zustandslos}

Diese Einschränkung bezieht sich auf die Interaktion zwischen Client und Server und ist ein fundamentales Prinzip, das vorschreibt, dass jede Anfrage des Clients an den Server alle notwendigen Informationen beinhalten muss, um diese unabhängig von vorherigen Interaktionen zu verarbeiten. 
Dieses Prinzip stellt sicher, dass der Server keine Zustandsinformationen speichern muss, wodurch die Abhängigkeit von einer Anfrage zur nächsten eliminiert wird.
Aus architektonischer Sicht bietet die Zustandslosigkeit signifikante Vorteile bezüglich Visibilität, Zuverlässigkeit und Skalierbarkeit.
Doch die Performanz kann darunter leider, da es zu mehrfachen Senden gleicher Informationen kommen kann. \cite{Fielding:2000:REST}


\subsection{Cache}

Der Cache-Constraint gibt vor, dass Antworten des Servers als cacheable oder non-cacheable gekennzeichnet werden. 
Dies ermöglicht es dem Client, Antworten für zukünftige Anfragen wiederzuverwenden, was potenziell zu erheblichen Leistungssteigerungen führen kann. Allerdings kann dies die Zuverlässigkeit beeinträchtigen, da die im Cache gespeicherten Daten möglicherweise nicht mehr mit den aktuellen Daten auf dem Server übereinstimmen. \cite{Fielding:2000:REST}


\subsection{Uniform Interface}

Die Haupteigenschaft vom REST-Architekturstil ist das Uniform Interface und wird durch vier untergeordnete Einschränkungen definiert, die gemeinsam das Konzept von Ressourcen und deren Repräsentationen prägen.

\textbf{{ Ressourcen und Repräsentationen}} 

Roy Fielding definiert eine Ressource als eine benennbare Information, die über eine zeitlich variable Funktion identifiziert wird. 
Repräsentationen beschreibt Fielding als eine Byte-Sequenz, ergänzt durch Metadaten, die diese Bytes beschreiben. Die Repräsentation ist das, was als Antwort auf eine Ressourcenanforderung übermittelt wird.
Diese entscheidende Trennung von Ressource und Repräsentation erleichtert die Verwaltung von Ressourcen, indem sie es ermöglicht, dass diese unabhängig von ihren spezifischen Repräsentationen identifiziert und manipuliert werden können \cite{Fielding:2000:REST}.


\textbf{{ Identifikation und Manipulation von Ressourcen}} 

Die REST-Architektur nutzt Uniform Resource Identifiers (URIs), um Ressourcen zu identifizieren. URIs vereinen die Funktionen von URLs (zur Lokalisierung) und URNs (zur Benennung), wodurch die Flexibilität im Umgang mit Ressourcen erhöht wird, da sie eine klare und eindeutige Identifikation ermöglichen. 
Die Identifikation durch URIs sollte sich nie auf spezifische Repräsentationen beziehen, sondern auf die konzeptuelle Einordnung der Ressource.
Also eine URI soll keine Dateiendung wie "path/file.json" enthalten, um flexibel zu bleiben.
Die Manipulation von Ressourcen erfolgt über ihre Repräsentationen, wobei die Ressource selbst unverändert bleibt. 
Dies unterstützt die Skalierbarkeit und Wartung von Webanwendungen, indem es Änderungen erleichtert, ohne dass bestehende Verweise angepasst werden müssen. \cite{Fielding:2000:REST}


\textbf{{ Selbstbeschreibende Nachrichten und HATEOAS}}

Selbstbeschreibende Nachrichten gewährleisten, dass jede Anfrage oder Antwort so wenig Kontext wie möglich benötigt. 
Durch die Verwendung standardisierter Methoden und explizite Angaben zur Cacheability in den Antworten fördert dies die Zustandslosigkeit von REST-Services.

Die Einschränkung Hypermedia as the Engine of Application State (HATEOAS) zielt darauf ab, eine maximale Entkopplung zwischen Client und Server zu gewährleisten. 
Sie erfordert, dass der Client keine Kenntnisse über die interne Struktur des Services benötigt, da alle Zustandsübergänge auf dem Server durch die Auswahl von durch den Server bereitgestellten Hyperlinks gesteuert werden. 
Dies ermöglicht eine dynamische Navigation und Interaktion mit dem Service, was die Flexibilität und Erweiterbarkeit des Webservices erhöht. \cite{Fielding:2000:REST}


\subsection{Layered System}

Das Konzept des geschichteten Systems (Layered System)ermöglicht eine hierarchische Schichtenarchitektur. 
Jede Komponente der REST Architektur interagiert nur mit der direkt angrenzenden Schicht, was die Komplexität des Systems reduziert und die Unabhängigkeit von der zugrunde liegenden Plattform erhöht. 
Geschichtete Systeme können auch zur Kapselung alter Dienste und zum Schutz neuer Dienste vor veralteten Clients genutzt werden und unterstützen die Skalierbarkeit durch Lastverteilung über mehrere Netzwerke.

Durch den zusätzlich entstehenden Overhead
\footnote{
	Als Overhead bezeichnet man jene Daten, die neben den eigentlichen Nutzdaten zusätzlich übertragen oder gespeichert werden, um die Kommunikation und Datenverarbeitung zu unterstützen.
}
und die dadurch verursachte Latenz, kann die Leistung beeinträchtigt werden.
Insgesamt fördert die Kombination aus geschichtetem System und uniformen Schnittstellen die Modularität und Wartbarkeit der Architektur, indem sie ein flexibles und effizientes Management von Diensten ermöglicht. \cite{Fielding:2000:REST}


\subsection{Code-On-Demand}

Code-On-Demand ist eine optionale Einschränkung innerhalb des REST Architekturstils. 
Dieses Prinzip erweitert die Flexibilität einer Client-Server-Interaktion, indem es Servern ermöglicht, ausführbaren Code an Clients zu senden, die diesen Code dann ausführen können. 
Somit könne Server nicht nur reine Daten, sondern auch Funktionalität in Form von ausführbaren Code, an den Client zurück senden. \cite{Fielding:2000:REST}


\subsection{Zusammenfassung der REST-Constraints}

Nur wenn all diese Constraints, abgesehen von Code-On-Demand, eingehalten werden, kann ein System als RESTful deklariert werden. 
In der folgenden Auflistung werden die einzelnen Einschränkungen noch einmal zur Übersicht aufgelistet und kurz beschrieben.

\begin{enumerate}
	\item \textbf{Client-Server}:   \\  Strikte Trennung von Client und Server Anliegen
	\item \textbf{Stateless}:       \\  Jede Client-Anfrage beinhaltet alle notwendigen Informationen
	\item \textbf{Cache}:           \\  Alle Server-Antworten mit (non-)cacheable deklarieren
	
	\item \textbf{Uniform Interface} 
	\begin{enumerate}[label*=\arabic*.]
		\item \textbf{Identification of Resources}: 					  \\ Jede Ressource hat einen eindeutigen Identifikator (URI)
		\item \textbf{Manipulation of Resources through Representations}: \\ Veränderungen ausschließlich über Repräsentationen
		\item \textbf{Self-descriptive Messages}: 						  \\ Jede Nachricht kann für sich selbst stehen
		\item \textbf{HATEOAS}: 										  \\ Dynamische Client-Interaktionen durch Hyperlinks in Serverantworten
	\end{enumerate}
		
	\item \textbf{Layered System}:  \\ Geschichtete und voneinander isolierte Architektur
	\item \textbf{Code-On-Demand}:  \\ Server können ausführbaren Code an Client senden
\end{enumerate}


\section{GraphQL}

Graph Query Language (GraphQL) ist eine Datenabfrage- und Datenmanipulation-Sprache für Application Programming Interfaces. 
Kern von GraphQL ist das Typensystem, das es ermöglicht, präzise die Struktur und die benötigten Daten zu definieren. Dies ermöglicht es mit einzelnen Abfragen Operationen in sehr verschachtelten Daten durchzuführen.
Eine weiteres Merkmal von GraphQL ist die Introspektionsfähigkeit, welche eine Selbstbeschreibung des Schemas ermöglicht.
Dies unterstützt die Entwicklung und Validierung von Client-Anfragen.


\subsection{Anwendungsbeispiel}

Im folgenden Abschnitt wird anhand eines Beispiels eine GraphQL \hyperref[alg:graphql-request]{Anfrage} und \hyperref[alg:graphql-response]{Antwort} illustriert.
Angenommen, wir haben eine Webanwendung mit Benutzer:innen, Postings und Kommentaren, welche in verschiedenen Datenbanken gespeichert sind.
Wollen wir nun für eine Person mit ID 42 alle Postings und Kommentare abrufen, genügt in GraphQL eine einzige Anfrage. \cite{Quinamera:2023:GraphMappingStudy}

\renewcommand{\lstlistingname}{Query}
\begin{lstlisting}[
		language=GraphQL, 
		caption={GraphQL-Anfrage}, 
		label={alg:graphql-request}
	]
		user(id: 42) {
			name
			email
			posts {
				title
				date
				comments {
					text
					date
				}
			}
		}
\end{lstlisting}

Die entsprechende Server-Antwort auf die Anfrage sieht wie folgt aus:

\begin{lstlisting}[
	language=GraphQL, 
	caption={GraphQL-Antwort}, 
	label={alg:graphql-response}
	]
	"user" {
		"name": "Zaphod",
		"email": "zaphod@mail.univer.se",
		"posts": [
			{
				"text": "Answer is...",
				"date": "2000-01-01",
				"comments": [
					{
						"text": "...42"
						"date": "2000-01-02"
					}
				]
			}
		]
	}
\end{lstlisting}

Betrachtet man die JSON-Antwort als Key-Value Paare, kann man erkennen, dass die Anfragen den Antworten entsprechen, nur ohne dem Value Part.


\section{Vergleich zwischen GraphQL und REST}

Der Großteil aller Web-Schnittstellen verwendet den REST-Architekturstil, auch wie bereits oben erläutert, nur ein geringer Anteil tatsächlich alle REST-Einschränkungen umsetzt.
Der zweite große, und immer mehr Anteil gewinnende, Player ist GraphQL.
In diesem Abschnitt werden die zwei Technologien miteinander verglichen und erwogen, welche der beiden die bessere Lösung für die benötigte Schnittstelle ist.


\subsection{Performanz}








Derzeit dominieren zwei Technologien die Web-API-Entwicklug: REST und GraphQL. Generell
sind Rest und GraphQL bei POST- und PUT-Abfragen performant ausgeglichen, jedoch bei
GET-Abfragen unterscheiden sie sich signifikant[4]. Diese Unterschiede lassen sich haupts¨achlich
darauf zur¨uckf¨uhren, dass REST-Implementierungen typischerweise mehrere festgelegte Endpunkte
ben¨otigen, um Daten aus verschiedenen Ressourcen abzurufen[3]. Im Gegensatz dazu
nutzt GraphQL einen einzigen, flexiblen Endpunkt, der komplexe Datenabfragen effizient verarbeiten
kann[3]. Diese Architektur erm¨oglicht es GraphQL, den Herausforderungen des Over- und
Underfetchings entgegenzuwirken, welche bei REST-basierten APIs h¨aufig auftreten [3]. Eine
Studie von Lawi et al.,[5] zeigte, dass REST bis zu 50% schneller als GraphQL antwortet und
37% h¨oheren Durchsatz erzielt. Allerdings GraphQL 37% weniger CPU-Rechenleistung und 39%
weniger Speicherbelegung beansprucht[5].
Eine im Zuge dieser Arbeit durchgef¨uhrten Anforderungsanalyse wird

4. rest vs graphql
\cite{Quinamera:2023:GraphMappingStudy}
\cite{Vohra:2022:GraphVsRestImplementation}
\cite{Lawi:2021:GraphVsRestPerformance}





%\begin{table}[h]
%	\centering
%	
%	\begin{tabular}{ll|l}
%		
%		1   & Client-Server       & Strikte Trennung von Client und Server Anliegen \\
%		\hline
%		2   & Zustandslos         & Jede Client-Anfrage beinhaltet alle notwendigen Informationen \\
%		\hline 
%		3   & Cache      		  & Alle Server-Antworten mit (non-)cacheable deklarieren \\
%		\hline
%		4   & Uniform Interface \\
%		4.1 & Ressources          & some text \\
%		4.2 & Representations     & some text \\
%		4.3 & Selbstbeschreibend  & some text \\
%		4.4 & HATEOAS             & some text \\
%		\hline
%		5   & Layered System      & some text \\
%		\hline
%		6   & Code-On-Demand      & some text \\
%		
%	\end{tabular}
%	
%	\caption{REST-Constraints}
%	\label{tab:REST-Constraints}
%\end{table}









\chapter{Methode}
\todo{Enter your text here.}



\chapter{Analyse und Design}
\todo{Enter your text here.}

\section{Anforderungsanalyse}

\section{Design \& Konzept der Web-Schnittstelle}



\chapter{Implementierung}
\todo{Enter your text here.}

\section{Prototypische Implementierung der Schnittstelle}



\chapter{Validierung der Lösung}
\todo{Enter your text here.}

\section{Vergleich mit ähnlichen Systemen}

\section{Experteninterviews - Durchführung}

\section{Experteninterviews - Analyse}



\chapter{Ergebnisse}
\todo{Enter your text here.}

\section{Ausblick}

\section{Zusammenfassung}


% Remove following line for the final thesis.
\input{intro.tex} % A short introduction to LaTeX.

\backmatter

% Use an optional list of figures.
\listoffigures % Starred version, i.e., \listoffigures*, removes the toc entry.

% Use an optional list of tables.
\cleardoublepage % Start list of tables on the next empty right hand page.
\listoftables % Starred version, i.e., \listoftables*, removes the toc entry.

% Use an optional list of alogrithms.
\listofalgorithms
\addcontentsline{toc}{chapter}{Liste der Algorithmen}

% Add an index.
\printindex

% Add a glossary.
\printglossaries

% Add a bibliography.
\bibliographystyle{IEEEtran}
\bibliography{thesis}

\end{document}
