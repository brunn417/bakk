% Copyright (C) 2014-2020 by Thomas Auzinger <thomas@auzinger.name>

\documentclass[draft,final]{vutinfth} % Remove option 'final' to obtain debug information.

% Load packages to allow in- and output of non-ASCII characters.
\usepackage{lmodern}        % Use an extension of the original Computer Modern font to minimize the use of bitmapped letters.
\usepackage[T1]{fontenc}    % Determines font encoding of the output. Font packages have to be included before this line.
\usepackage[utf8]{inputenc} % Determines encoding of the input. All input files have to use UTF8 encoding.

% Extended LaTeX functionality is enables by including packages with \usepackage{...}.
\usepackage{csquotes}   % used for easy quotes
\usepackage{amsmath}    % Extended typesetting of mathematical expression.
\usepackage{amssymb}    % Provides a multitude of mathematical symbols.
\usepackage{mathtools}  % Further extensions of mathematical typesetting.
\usepackage{microtype}  % Small-scale typographic enhancements.
\usepackage[inline]{enumitem} % User control over the layout of lists (itemize, enumerate, description).
\usepackage{multirow}   % Allows table elements to span several rows.
\usepackage{booktabs}   % Improves the typesettings of tables.
\usepackage{subcaption} % Allows the use of subfigures and enables their referencing.
\usepackage[ruled,linesnumbered,algochapter]{algorithm2e} % Enables the writing of pseudo code.
\usepackage[usenames,dvipsnames,table]{xcolor} % Allows the definition and use of colors. This package has to be included before tikz.
\usepackage{nag}       % Issues warnings when best practices in writing LaTeX documents are violated.
\usepackage{todonotes} % Provides tooltip-like todo notes.
\usepackage{hyperref}  % Enables cross linking in the electronic document version. This package has to be included second to last.
\usepackage[acronym,toc]{glossaries} % Enables the generation of glossaries and lists fo acronyms. This package has to be included last.

% Define convenience functions to use the author name and the thesis title in the PDF document properties.
\newcommand{\authorname}{Paul Brunner} % The author name without titles.
\newcommand{\thesistitle}{} % The title of the thesis. The English version should be used, if it exists.

% Set PDF document properties
\hypersetup{
    pdfpagelayout   = TwoPageRight,           % How the document is shown in PDF viewers (optional).
    linkbordercolor = {Melon},                % The color of the borders of boxes around crosslinks (optional).
    pdfauthor       = {\authorname},          % The author's name in the document properties (optional).
    pdftitle        = {\thesistitle},         % The document's title in the document properties (optional).
    pdfsubject      = {Subject},              % The document's subject in the document properties (optional).
    pdfkeywords     = {a, list, of, keywords} % The document's keywords in the document properties (optional).
}

\setpnumwidth{2.5em}        % Avoid overfull hboxes in the table of contents (see memoir manual).
\setsecnumdepth{subsection} % Enumerate subsections.

\nonzeroparskip             % Create space between paragraphs (optional).
\setlength{\parindent}{0pt} % Remove paragraph identation (optional).

\makeindex      % Use an optional index.
\makeglossaries % Use an optional glossary.
%\glstocfalse   % Remove the glossaries from the table of contents.

% Set persons with 4 arguments:
%  {title before name}{name}{title after name}{gender}
%  where both titles are optional (i.e. can be given as empty brackets {}).
\setauthor{}{\authorname}{}{}
\setauthorextra
\setadvisor{Dipl.-Ing.}{Thomas Artner}{}{male}

% For bachelor and master theses:
%\setfirstassistant{Pretitle}{Forename Surname}{Posttitle}{male}
%\setsecondassistant{Pretitle}{Forename Surname}{Posttitle}{male}
%\setthirdassistant{Pretitle}{Forename Surname}{Posttitle}{male}

% For dissertations:
\setfirstreviewer{Pretitle}{Forename Surname}{Posttitle}{male}
\setsecondreviewer{Pretitle}{Forename Surname}{Posttitle}{male}

% For dissertations at the PhD School and optionally for dissertations:
\setsecondadvisor{Pretitle}{Forename Surname}{Posttitle}{male} % Comment to remove.

% Required data.
\setregnumber{11919163}
\setdate{01}{06}{2024} % Set date with 3 arguments: {day}{month}{year}.
\settitle{\thesistitle}{Die Evaluierung und Implementierung einer Web-API} % Sets English and German version of the title (both can be English or German). If your title contains commas, enclose it with additional curvy brackets (i.e., {{your title}}) or define it as a macro as done with \thesistitle.
\setsubtitle{Optional Subtitle of the Thesis}{Technologieevaluierung und Implementierung einer Web-
		Schnittstelle für Gebäudezertifizierungsdaten:
		Eine Fallstudie bei der Pulswerk GmbH.} % Sets English and German version of the subtitle (both can be English or German).

% Select the thesis type: bachelor / master / doctor / phd-school.
% Bachelor:
\setthesis{bachelor}
%
% Master:
%\setthesis{master}
%\setmasterdegree{dipl.} % dipl. / rer.nat. / rer.soc.oec. / master
%
% Doctor:
%\setthesis{doctor}
%\setdoctordegree{rer.soc.oec.}% rer.nat. / techn. / rer.soc.oec.
%
% Doctor at the PhD School
%\setthesis{phd-school} % Deactivate non-English title pages (see below)

% For bachelor and master:
\setcurriculum{Media Informatics and Visual Computing}{Software \& Information Engineering} % Sets the English and German name of the curriculum.

% For dissertations at the PhD School:
\setfirstreviewerdata{Affiliation, Country}
\setsecondreviewerdata{Affiliation, Country}


\begin{document}

\frontmatter % Switches to roman numbering.
% The structure of the thesis has to conform to the guidelines at
%  https://informatics.tuwien.ac.at/study-services

%\addtitlepage{naustrian} % German title page (not for dissertations at the PhD School).
% \addtitlepage{english} % English title page.
\addinsotitlepage{naustrian}
\addstatementpage

\begin{danksagung*}
\todo{Ihr Text hier.}
danke mutti
\end{danksagung*}

\begin{acknowledgements*}
\todo{Enter your text here.}
wtf ist das
\end{acknowledgements*}

\begin{kurzfassung}
\todo{Ihr Text hier.}
Dieses Template dient als Vorlage für die Erstellung einer Diplomarbeit am INSO. Individuelle Erweiterungen, Strukturanpassungen und Layout-Veränderungen können und sollen selbstverständlich nach persönlichem Ermessen und in Rücksprache mit Ihrem Betreuer vorgenommen werden. 

Diplomarbeiten aus Informatik können in deutscher oder englischer Sprache verfasst werden, Arbeiten aus Business Informatics müssen auf Englisch geschrieben werden.

Die Kurzfassung ist der Teil der Arbeit, der wohl am häufigsten gelesen wird – so wird sie beispielsweise im Epilog-Band der Fakultät publiziert und einem breiten Publikum verfügbar gemacht. Empfohlen wird, die Kurzfassung erst nach Finalisierung der gesamten Arbeit zu schreiben.

Aufbau: In der Kurzfassung werden auf einer 3/4 bis maximal einer Seite die Kernaussagen der Diplomarbeit zusammengefasst. Dabei sollte zunächst die Motivation/ der Kontext der vorliegenden Arbeit dargestellt werden, und dann kurz die Frage-/ Problemstellung erläutert werden, max. 1 Absatz! Im nächsten Absatz auf die Methode/ Verfahrensweise/ das konkrete Fallbeispiel eingehen, mit deren Hilfe die Ergebnisse erzielt wurden. Im Zentrum der Kurzfassung stehen die zentralen eigenen Ergebnisse der Arbeit, die den Wert der vorliegenden wissenschaftlichen Arbeit ausmachen. Hier auch, wenn vorhanden, eigene Publikationen erwähnen.

Wichtig: Verständlichkeit! Abkürzungen immer zuerst ausschreiben, in Klammer die Erklärung: Im Rahmen der vorliegenden Arbeit werden Non Governmental-Organisationen (NGOs) behandelt, \ldots

Bei theoretischen Diplomarbeiten, z.B. Literaturüberblick und Grundlagen zu einem größeren Themenblock, sollte in der Kurzfassung deutlich der Bedarf an einer solchen Übersicht und der Nutzen für die akademische Gemeinschaft aufgezeigt werden.


\textbf{Keywords:} \emph{5 – max. 8 Keywords zur Arbeit eingeben}
\end{kurzfassung}

\begin{abstract}
\todo{Enter your text here.}
Hier werden auf einer halben bis maximal einer Seite die Kernaussagen der Diplomarbeit in Englisch zusammengefasst ( = Übersetzung der Kurzfassung, am besten von einem \enquote{Native Speaker} Korrektur lesen lassen). Englischer Abstract ist auch bei auf Deutsch geschriebenen Arbeiten verpflichtend von der Fakultät vorgesehen.


\textbf{Keywords:} \emph{Übersetzung der deutschen Keywords}
\end{abstract}

% Select the language of the thesis, e.g., english or naustrian.
\selectlanguage{naustrian}

% Add a table of contents (toc).
\tableofcontents % Starred version, i.e., \tableofcontents*, removes the self-entry.

% Switch to arabic numbering and start the enumeration of chapters in the table of content.
\mainmatter


\chapter{EINLEITUNG}

\section{Hintergrund und Motivation}

Die Pulswerk GmbH ist seit mehr als 2 Jahren mein Arbeitgeber.
Ihr Tätigkeitsbereich ist sehr umfassend, aber einer der Kernbereiche ist die Bereitstellung von Web-Applikationen zur Zertifizierung von Gebäudedaten.
Für verschiedene Zertifikate und Kunden gibt es unterschiedliche Plattformen und Kriteriensets, jedoch ist eine generische Grundstruktur die Basis. 
Mein derzeitiger Tätigkeitsbereich ist die Qualitätssicherung und Unterstützung bei der Weiterentwicklung dieser Plattformen.
Diese Gebäude-spezifischen Daten sind auf dem Webserver der Pulswerk GmbH gespeichert und nur über diese Formulare und bestimmten Webseiten zugänglich.
Damit ein strukturierter und übersichtlicher Zugang zu diesen Daten ermöglicht wird, soll eine Web-Schnittstelle in Form eines Application Programming Interface (API) implementiert werden.

Im Zuge dieser Bachelorarbeit für Informatik, die zum Abschluss des Bachelorstudiums Software \& Information 033 534 Engineering an der Technische Universität Wien Voraussetzung ist, werde ich diese Problemstellung behandeln und zu einer Lösung beitragen.
 
 
\section{Zielsetzung der Arbeit}

Da es in der Zukunft viele unterschiedliche Benutzergruppen dieser Schnittstelle geben soll, wird eine generische, aber auf die Anforderungen abgestimmte, Implementierung erwartet. 
Um dies zu ermöglichen, muss im Vorfeld der Implementierung eine genaue Evaluierung der verschiedenen Technologien und Ansprüche durchgeführt werden.
 
Das Ziel dieser Arbeit ist es, den derzeitigen Stand der Technik in Bezug auf Web-Schnittstellen zu recherchieren und auf dem erlangten Wissen aufbauend eine API entwerfen.
Diese soll nicht nur den Ansprüchen der Stakeholder gerecht werden, sondern auch technische nicht funktionalen Anforderungen sättigen.
Dabei sind Aspekte wie Sicherheit, Antwortzeit, Benutzerfreundlichkeit und Skalierbarkeit grundlegende Kriterien, aber auch Parameter wie Over- und Underfetching
\footnote{
	Underfetching tritt auf, wenn eine Schnittstellenabfrage nicht alle erwünschten Daten in der Antwort enthält wohingegen beim Overfetching mehr als das Angeforderte retourniert wird.
} 
sind von Bedeutung und werden bei der Anforderungsanalyse berücksichtigt. 

Mittels Continuous Deployment soll eine nahtlose Integration in die bestehende Systemlandschaft der Pulswerk GmbH ermöglicht werden. 
Dies gewährleistet, dass die Web-Schnittstelle nicht nur den momentanen Anforderungen gerecht wird, sondern auch entsprechend flexibel ist, um sich an zukünftige Entwicklungen anzupassen.


\section{Struktur der Arbeit}

Zum Beginn der Arbeit werden mögliche Anwendungsfälle simuliert und in Zusammenarbeit mit den Stakeholder eine Anforderungsanalyse durchgeführt. 
Dabei werden die funktionalen und nicht-funktionalen Anforderungen untersucht und festgelegt.
Die daraus resultierenden Leistungsanforderungen liefern das Grundkonzept für die Schnittstelle. 

Im nächsten Schritt wird eine umfassende Recherche des aktuellen Stands der Technik und dazugehörenden Literatur durchgeführt. 
Hier liegt der Fokus auf der Auswahl welches Entwurfsmuster für diesen Anwendungsfall herangezogen werden soll und mithilfe welcher Technologien die Implementierung umzusetzen ist. 

Nach der Technologieevaluierung wird eine prototypische Implementierung gemacht. 
Dabei werden relevante Technologien und Frameworks entsprechend den Projektanforderungen ausgewählt. 
Dieser Prototyp dient als Grundlage für die iterative Verbesserung und Entwicklung einer vollständigen Web-Schnittstelle.

Nach der prototypischen Implementierung wird abschließend eine Validierung und Einschätzung der entwickelten Web-Schnittstelle durch Experteninterviews durchgeführt. 
Ziel ist es, Fachwissen und Erfahrung externer und interner Experten zu nutzen, um die entwickelte Lösung kritisch zu bewerten und mögliche Schwachstellen oder Optimierungspotenziale aufzudecken.



\chapter{GRUNDLAGEN}


\section{Beschreibung des Unternehmens und der Webapplikationen}

Die Pulswerk GmbH, ein Tochterunternehmen des Österreichischen Ökologie-Instituts, hat sich seit ihrer Gründung 1985 als ein Akteur im Bereich der ökologischen und nachhaltigen Entwicklung etabliert. 
Das Unternehmen tritt in einer breiten Palette von Sektoren als Initiator und Mitgestalter für zukunftsfähige Projekte auf und engagiert sich in diversen Netzwerken für die Planung und Realisierung nachhaltiger Initiativen.

Ein spezifischer Fokus der Pulswerk GmbH liegt auf dem Baugewerbe, insbesondere in Bezug auf Herausforderungen des Klimaschutzes. 
Sie bietet einige Webapplikationen an, die es ermöglichen, Gebäude nach verschiedenen Richtlinien wie klimaaktiv
\footnote{
	klimaaktiv ist ein österreichisches Programm zur Förderung klimafreundlichen Bauens, unterstützt durch das Bundesministerium für Klimaschutz. 
	Es bietet unter Anderem Zertifizierungen für energieeffiziente und nachhaltige Gebäude.
} 
oder EU-Taxonomie
\footnote{
	Die EU-Taxonomie ist ein EU-Klassifikationssystem, das festlegt, welche Investitionen als ökologisch nachhaltig
	gelten, um grüne Investitionen zu unterstützen und den Klimaschutz voranzutreiben.
	Die EU-Taxonomie bezieht sich auf alle Wirtschaftssektoren, wobei sich die Pulswerk GmbH mit dem Bausektor befasst. 
} 
einfach und ohne bürokratischen Aufwand zu zertifizieren. 

Diese Kataloge, die als umfangreiche Formulare mit diversen Eingabeoptionen wie Textfeldern, Checkboxen, Radio-Buttons und Upload-Funktionen konzipiert sind, erlauben Nutzern, Daten einzutragen und abhängig von Plattform und spezifischem Kriterienkatalog Punkte zu erhalten.
Innerhalb der verschiedenen Plattformen kommen unterschiedliche Kriterienkataloge zum Einsatz, die sich nicht nur zwischen den einzelnen Webapplikationen unterscheiden, sondern auch innerhalb einer Plattform abhängig von der Version, dem Gebäudetyp und anderen Faktoren variieren können. 
Zusätzlich ist es möglich, sofern es konfiguriert wurde, Projekte auch zwischen verschiedenen Kriterienkataloge zu verschieben, und die übereinstimmenden Felder werden übernommen. 
Somit können Projekte gleichzeitig für mehrere Zertifizierungen deklariert und eingereicht werden.

Sobald ein Projekt alle Mindestanforderungen erfüllt und die Eingaben komplettiert sind, kann es zur Plausibilitätsprüfung eingereicht werden. 
Diese Prüfung wird von internen oder externen Auditoren vorgenommen und kann sich über mehrere Durchläufe erstrecken. 
Nach erfolgreichem Abschluss der Überprüfung besteht die Option, das zertifizierte Gebäude auf verschiedenen Webseiten visuell ansprechend aufbereitet zu präsentieren und damit für die Öffentlichkeit zugänglich zu machen.


\section{Erfordernis einer API für den internen und externen Datenzugriff}

In der aktuellen Dateninfrastruktur der Plattformen befinden sich über 10.000 Gebäudedatensätze, die entweder noch bearbeitet oder bereits finalisiert sind. 
Die dezentrale Verteilung dieser Datensätze schränkt jedoch die Möglichkeit eines kontrollierten Zugriffs ein. 
Daraus ergibt sich die Notwendigkeit einer Schnittstelle, die in der Lage ist, effizient durch diese umfangreichen Datenbestände zu navigieren und die geforderten Informationen abzurufen. 
Diese Schnittstelle ermöglicht somit einen systematischen und strukturierten Zugang zu den in Datenbanken auf dem Server gespeicherten Datensätzen und trägt damit zur Optimierung der Datenverwaltung und Datennutzung bei.

Von solch einer Schnittstelle profitieren sowohl interne Teams als auch externe Kunden. 
Für die internen Operationen der Pulswerk GmbH ermöglicht die Schnittstelle eine vertiefte Analyse und Evaluation der durchgeführten Arbeiten, was zu einer verbesserten Effizienz und Effektivität der internen Prozesse beitragen kann. 
Externe Kunden erhalten durch die API einen detaillierten Einblick in ihre individuellen Beiträge und den jeweiligen Fortschritt in Bezug auf Klimaschutzmaßnahmen. 
Zusätzlich unterstützt die Integration der API eine nahtlose Kommunikation mit bestehenden Abrechnungssystemen, wie z.B. SAP
\footnote{
	Systems, Applications \& Products in Data Processing ist ein weltweit führendes Unternehmen im Bereich Unternehmenssoftware. 
	Diese Software unterstützt Firmen bei der Verwaltung und Optimierung ihrer Geschäftsprozesse.
}
, was eine Vereinfachung der Geschäfts- und Abrechnungsprozesse zur Folge hat.
Den Kunden wird über diese Plattform Zugang gewährt, wobei der Zugriff durch entsprechende Berechtigungen reguliert wird. 
Die Entwicklung einer generischen API, die diesen Anforderungen gerecht wird, sowie die Erstellung einer nachvollziehbaren Dokumentation bildet die essentielle Basis, um Benutzerfreundlichkeit des Systems sicherzustellen.


\section{Theoretische Konzepte von Web-Schnittstellen}

APIs sind essentielle Mechanismen, die Interaktionen zwischen zwei Softwarekomponenten ermöglichen. 
Beide Seiten verwenden dafür bestimmte Schnittstellenspezifikationen und Protokolle, um einen nahtlosen Datenaustausch zu gewährleisten, oder im Fehlerfall aussagekräftige Fehlermeldungen bereitzustellen.
Es existiert eine Vielzahl an Architekturen solcher Schnittstellen, die je nach Anwendungsfall eingesetzt werden.
Im Grundsatz sendet eine Client-Applikation eine Anfrage, die über ein Netzwerk an den Server gerichtet wird. 
Dieser interpretiert die Anfrage gemäß der API-Konventionen, verarbeitet diese, und antwortet dem Client mit den angefragten Daten.

Web-Schnittstellen repräsentieren eine spezialisierte Kategorie von APIs, die spezifisch für eine Kommunikation zwischen Web-Clients –- üblicherweise Web-Browsern –- und Web-Servern ausgelegt sind. 
Hierbei wird der Datenaustausch über das Hypertext Transfer Protocol (HTTP) oder dessen sichere Variante, HTTPS, abgewickelt, wobei eine große Bandbreite an Datenformaten wie HTML, XML oder JSON zum Einsatz kommen kann.


\section{Designprinzipien von Web-Schnittstellen}

Das Design von APIs unterliegen einer Vielzahl von Ansätzen, deren übergreifendes Ziel es ist, eine robuste und intuitiv bedienbare Schnittstelle zu entwickeln. 
Im Zentrum steht der Entwurf einer API, die nicht nur in ihrer Lebensdauer beständig, sondern auch wartungsarm ist. 
Eine effektive API zeichnet sich durch ihre Benutzerfreundlichkeit aus und bedarf keiner umfangreichen Dokumentation, da sie weitgehend selbsterklärend sein sollte. \cite{How2Design_good_API}

Im Vorfeld der Entwicklung einer solchen Schnittstelle ist eine umfassende Anforderungsanalyse essenziell, um eine adäquate Erfüllung der spezifischen Ansprüche zu gewährleisten. 
Mit der Weiterentwicklung der Anwendungsfälle muss auch die API in ihrer Funktionalität flexibel und adaptiv gestaltet sein, um zukünftigen Anforderungen gerecht zu werden. 
Dies bedingt, dass die Weiterentwicklung der API stets unter Berücksichtigung dieser dynamischen Entwicklung erfolgt. \cite{How2Design_good_API}

Des Weiteren spielen aussagekräftige Beispiele für die Nutzung der API eine grundlegende Rolle, um deren Leistungsfähigkeit und Anwendbarkeit zu veranschaulichen. 
Die Klarheit in der Namensgebung ist ein weiterer kritischer Faktor, der die Zugänglichkeit und Verständlichkeit der API maßgeblich beeinflusst. Darüber hinaus ist es vorteilhaft, wenn alle Anfragen im String-Format verarbeitet werden können und die API konsistente sowie typgerechte Rückgabeformate liefert, um die Datenverarbeitung und Datenintegration zu vereinfachen. \cite{How2Design_good_API}


\section{Anforderungen an die Schnittstelle}



\chapter{Ähnliche Arbeiten}
\todo{Enter your text here.}

\section{Literaturüberblick zu vorhandenen Forschungsarbeiten im Bereich Web-Schnittstellen}



\chapter{Methode}
\todo{Enter your text here.}



\chapter{Analyse und Design}
\todo{Enter your text here.}

\section{Anforderungsanalyse}

\section{Design \& Konzept der Web-Schnittstelle}



\chapter{Implementierung}
\todo{Enter your text here.}

\section{Prototypische Implementierung der Schnittstelle}



\chapter{Validierung der Lösung}
\todo{Enter your text here.}

\section{Vergleich mit ähnlichen Systemen}

\section{Experteninterviews - Durchführung}

\section{Experteninterviews - Analyse}



\chapter{Ergebnisse}
\todo{Enter your text here.}

\section{Ausblick}

\section{Zusammenfassung}


% Remove following line for the final thesis.
\input{intro.tex} % A short introduction to LaTeX.

\backmatter

% Use an optional list of figures.
\listoffigures % Starred version, i.e., \listoffigures*, removes the toc entry.

% Use an optional list of tables.
\cleardoublepage % Start list of tables on the next empty right hand page.
\listoftables % Starred version, i.e., \listoftables*, removes the toc entry.

% Use an optional list of alogrithms.
\listofalgorithms
\addcontentsline{toc}{chapter}{Liste der Algorithmen}

% Add an index.
\printindex

% Add a glossary.
\printglossaries

% Add a bibliography.
\bibliographystyle{IEEEtran}
\bibliography{intro}

\end{document}
